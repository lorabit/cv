%%%%%%%%%%%%%%%%%%%%%%%%%%%%%%%%%%%%%%%%%%%%%%%%%%%%%%%%%%%%%%%%%%%%%%%%
%%%%%%%%%%%%%%%%%%%%%% Simple LaTeX CV Template %%%%%%%%%%%%%%%%%%%%%%%%
%%%%%%%%%%%%%%%%%%%%%%%%%%%%%%%%%%%%%%%%%%%%%%%%%%%%%%%%%%%%%%%%%%%%%%%%

%%%%%%%%%%%%%%%%%%%%%%%%%%%%%%%%%%%%%%%%%%%%%%%%%%%%%%%%%%%%%%%%%%%%%%%%
%% NOTE: If you find that it says                                     %%
%%                                                                    %%
%%                           1 of ??                                  %%
%%                                                                    %%
%% at the bottom of your first page, this means that the AUX file     %%
%% was not available when you ran LaTeX on this source. Simply RERUN  %%
%% LaTeX to get the ``??'' replaced with the number of the last page  %%
%% of the document. The AUX file will be generated on the first run   %%
%% of LaTeX and used on the second run to fill in all of the          %%
%% references.                                                        %%
%%%%%%%%%%%%%%%%%%%%%%%%%%%%%%%%%%%%%%%%%%%%%%%%%%%%%%%%%%%%%%%%%%%%%%%%

%%%%%%%%%%%%%%%%%%%%%%%%%%%% Document Setup %%%%%%%%%%%%%%%%%%%%%%%%%%%%

% Don't like 10pt? Try 11pt or 12pt
\documentclass[10pt]{article}

% The automated optical recognition software used to digitize resume
% information works best with fonts that do not have serifs. This
% command uses a sans serif font throughout. Uncomment both lines (or at
% least the second) to restore a Roman font (i.e., a font with serifs).
%\usepackage{times}
%\renewcommand{\familydefault}{\sfdefault}

% This is a helpful package that puts math inside length specifications
\usepackage{calc}
\usepackage{comment}

% Simpler bibsection for CV sections
% (thanks to natbib for inspiration)
\makeatletter
\newlength{\bibhang}
\setlength{\bibhang}{1em} %1em}
\newlength{\bibsep}
 {\@listi \global\bibsep\itemsep \global\advance\bibsep by\parsep}
\newenvironment{bibsection}%
        {\begin{enumerate}{}{%
%        {\begin{list}{}{%
       \setlength{\leftmargin}{\bibhang}%
       \setlength{\itemindent}{-\leftmargin}%
       \setlength{\itemsep}{\bibsep}%
       \setlength{\parsep}{\z@}%
        \setlength{\partopsep}{0pt}%
        \setlength{\topsep}{0pt}}}
        {\end{enumerate}\vspace{-.6\baselineskip}}
%        {\end{list}\vspace{-.6\baselineskip}}
\makeatother

% Layout: Puts the section titles on left side of page
\reversemarginpar

%
%         PAPER SIZE, PAGE NUMBER, AND DOCUMENT LAYOUT NOTES:
%
% The next \usepackage line changes the layout for CV style section
% headings as marginal notes. It also sets up the paper size as either
% letter or A4. By default, letter was used. If A4 paper is desired,
% comment out the letterpaper lines and uncomment the a4paper lines.
%
% As you can see, the margin widths and section title widths can be
% easily adjusted.
%
% ALSO: Notice that the includefoot option can be commented OUT in order
% to put the PAGE NUMBER *IN* the bottom margin. This will make the
% effective text area larger.
%
% IF YOU WISH TO REMOVE THE ``of LASTPAGE'' next to each page number,
% see the note about the +LP and -LP lines below. Comment out the +LP
% and uncomment the -LP.
%
% IF YOU WISH TO REMOVE PAGE NUMBERS, be sure that the includefoot line
% is uncommented and ALSO uncomment the \pagestyle{empty} a few lines
% below.
%

%% Use these lines for letter-sized paper
\usepackage[paper=letterpaper,
            %includefoot, % Uncomment to put page number above margin
            marginparwidth=1.2in,     % Length of section titles
            marginparsep=.05in,       % Space between titles and text
            margin=1in,               % 1 inch margins
            includemp]{geometry}

%% Use these lines for A4-sized paper
%\usepackage[paper=a4paper,
%            %includefoot, % Uncomment to put page number above margin
%            marginparwidth=30.5mm,    % Length of section titles
%            marginparsep=1.5mm,       % Space between titles and text
%            margin=25mm,              % 25mm margins
%            includemp]{geometry}

%% More layout: Get rid of indenting throughout entire document
\setlength{\parindent}{0in}

\usepackage[shortlabels]{enumitem}

%% Reference the last page in the page number
%
% NOTE: comment the +LP line and uncomment the -LP line to have page
%       numbers without the ``of ##'' last page reference)
%
% NOTE: uncomment the \pagestyle{empty} line to get rid of all page
%       numbers (make sure includefoot is commented out above)
%
\usepackage{fancyhdr,lastpage}
\pagestyle{fancy}
%\pagestyle{empty}      % Uncomment this to get rid of page numbers
\fancyhf{}\renewcommand{\headrulewidth}{0pt}
\fancyfootoffset{\marginparsep+\marginparwidth}
\newlength{\footpageshift}
\setlength{\footpageshift}
          {0.5\textwidth+0.5\marginparsep+0.5\marginparwidth-2in}
\lfoot{\hspace{\footpageshift}%
       \parbox{4in}{\, \hfill %
                    \arabic{page} of \protect\pageref*{LastPage} % +LP
%                    \arabic{page}                               % -LP
                    \hfill \,}}

% Finally, give us PDF bookmarks
\usepackage{color,hyperref}
\definecolor{darkblue}{rgb}{0.0,0.0,0.3}
\hypersetup{colorlinks,breaklinks,
            linkcolor=darkblue,urlcolor=darkblue,
            anchorcolor=darkblue,citecolor=darkblue}

%%%%%%%%%%%%%%%%%%%%%%%% End Document Setup %%%%%%%%%%%%%%%%%%%%%%%%%%%%


%%%%%%%%%%%%%%%%%%%%%%%%%%% Helper Commands %%%%%%%%%%%%%%%%%%%%%%%%%%%%

% The title (name) with a horizontal rule under it
% (optional argument typesets an object right-justified across from name
%  as well)
%
% Usage: \makeheading{name}
%        OR
%        \makeheading[right_object]{name}
%
% Place at top of document. It should be the first thing.
% If ``right_object'' is provided in the square-braced optional
% argument, it will be right justified on the same line as ``name'' at
% the top of the CV. For example:
%
%       \makeheading[\emph{Curriculum vitae}]{Your Name}
%
% will put an emphasized ``Curriculum vitae'' at the top of the document
% as a title. Likewise, a picture could be included:
%
%   \makeheading[\includegraphics[height=1.5in]{my_picutre}]{Your Name}
%
% the picture will be flush right across from the name.
\newcommand{\makeheading}[2][]%
        {\hspace*{-\marginparsep minus \marginparwidth}%
         \begin{minipage}[t]{\textwidth+\marginparwidth+\marginparsep}%
             {\large \bfseries #2 \hfill #1}\\[-0.15\baselineskip]%
                 \rule{\columnwidth}{1pt}%
         \end{minipage}}

% The section headings
%
% Usage: \section{section name}
\renewcommand{\section}[1]{\pagebreak[3]%
    \hyphenpenalty=10000%
    \vspace{1.3\baselineskip}%
    \phantomsection\addcontentsline{toc}{section}{#1}%
    \noindent\llap{\scshape\smash{\parbox[t]{\marginparwidth}{\raggedright #1}}}%
    \vspace{-\baselineskip}\par}

% An itemize-style list with lots of space between items
\newenvironment{outerlist}[1][\enskip\textbullet]%
        {\begin{itemize}[#1,leftmargin=*]}{\end{itemize}%
         \vspace{-.6\baselineskip}}

% An environment IDENTICAL to outerlist that has better pre-list spacing
% when used as the first thing in a \section
\newenvironment{lonelist}[1][\enskip\textbullet]%
        {\begin{list}{#1}{%
        \setlength{\partopsep}{0pt}%
        \setlength{\topsep}{0pt}}}
        {\end{list}\vspace{-.6\baselineskip}}

% An itemize-style list with little space between items
\newenvironment{innerlist}[1][\enskip\textbullet]%
        {\begin{itemize}[#1,leftmargin=*,parsep=0pt,itemsep=0pt,topsep=0pt,partopsep=0pt]}
        {\end{itemize}}

% An environment IDENTICAL to innerlist that has better pre-list spacing
% when used as the first thing in a \section
\newenvironment{loneinnerlist}[1][\enskip\textbullet]%
        {\begin{itemize}[#1,leftmargin=*,parsep=0pt,itemsep=0pt,topsep=0pt,partopsep=0pt]}
        {\end{itemize}\vspace{-.6\baselineskip}}

% To add some paragraph space between lines.
% This also tells LaTeX to preferably break a page on one of these gaps
% if there is a needed pagebreak nearby.
\newcommand{\blankline}{\quad\pagebreak[3]}
\newcommand{\halfblankline}{\quad\vspace{-0.5\baselineskip}\pagebreak[3]}

% Uses hyperref to link DOI
\newcommand\doilink[1]{\href{http://dx.doi.org/#1}{#1}}
\newcommand\doi[1]{doi:\doilink{#1}}

% For \url{SOME_URL}, links SOME_URL to the url SOME_URL
\providecommand*\url[1]{\href{#1}{#1}}
% Same as above, but pretty-prints SOME_URL in teletype fixed-width font
\renewcommand*\url[1]{\href{#1}{\texttt{#1}}}

% For \email{ADDRESS}, links ADDRESS to the url mailto:ADDRESS
\providecommand*\email[1]{\href{mailto:#1}{#1}}
% Same as above, but pretty-prints ADDRESS in teletype fixed-width font
%\renewcommand*\email[1]{\href{mailto:#1}{\texttt{#1}}}

%\providecommand\BibTeX{{\rm B\kern-.05em{\sc i\kern-.025em b}\kern-.08em
%    T\kern-.1667em\lower.7ex\hbox{E}\kern-.125emX}}
%\providecommand\BibTeX{{\rm B\kern-.05em{\sc i\kern-.025em b}\kern-.08em
%    \TeX}}
\providecommand\BibTeX{{B\kern-.05em{\sc i\kern-.025em b}\kern-.08em
    \TeX}}
\providecommand\Matlab{\textsc{Matlab}}

%%%%%%%%%%%%%%%%%%%%%%%% End Helper Commands %%%%%%%%%%%%%%%%%%%%%%%%%%%

%%%%%%%%%%%%%%%%%%%%%%%%% Begin CV Document %%%%%%%%%%%%%%%%%%%%%%%%%%%%

\begin{document}
\makeheading{Yanan Xie}

\section{Contact Information}

% NOTE: Mind where the & separators and \\ breaks are in the following
%       table.
%
% ALSO: \rcollength is the width of the right column of the table
%       (adjust it to your liking; default is 1.85in).
%
\newlength{\rcollength}\setlength{\rcollength}{1.4in}%
%
\begin{tabular}[t]{@{}p{\textwidth-\rcollength}p{\rcollength}}
%\href{http://www.cse.osu.edu/}%
%     {Department of Computer Science and Engineering} & \\
%\href{http://www.osu.edu/}{The Ohio State University}
Post office 5, Alibaba Group, No. 969 West Wenyi Road,   & +87-18768117383 \\
Yu Hang District, Hangzhou 311121, China     & \email{lorabit@126.com}\\
\end{tabular}

%\section{Objective}

%Insert text here if you want to
%\begin{innerlist}
%\item More information and auxiliary documents can be found at\\\url{http://www.tedpavlic.com/facjobsearch/}
%\end{innerlist}

\section{Research Interests}

Information retrieval, recommender system, service computing, text mining, database

\section{Education}

\href{http://www.zju.edu.cn}{\textbf{Zhejiang University}},
Hangzhou, China
\begin{outerlist}
\item[] B.S.,
        \href{http://www.cs.zju.edu.cn/}
             {Computer Science},
             June 2014
        \begin{innerlist}
        \item Overall GPA: 3.17, Major GPA: 3.48
        \item Advisors:
              \href{http://mypage.zju.edu.cn/wuzhaohui}
                   {Zhaohui Wu, Ph.D} and
              \href{http://mypage.zju.edu.cn/0004274/0.html}
                   {Jian Wu, Ph.D}
        \end{innerlist}
\end{outerlist}



\section{Work Experience}

\textbf{Software Development Engineer in Test} \hfill {July 2014 to Present}
\begin{outerlist}

\item[] Taobao (China) Software CO.,LTD, Alibaba Group,\\
        Hangzhou, China\\
        Supervisor: Zhongjie Li, Ph.D.
        \begin{innerlist}
        \item Design test frameworks and quality monitors measuring search quality for all Alibaba products 
        \item Process QA-related data with Alibaba's Open Data Processing Service(ODPS)
        \end{innerlist}
\end{outerlist}


\section{Research Experience}

\textbf{Research Assistant} \hfill {Dec 2010 to June 2014}
\begin{outerlist}

\item[] CCNT lab,\\
        Zhejiang University\\
        Supervisors: Zhaohui Wu, Ph.D and Jian Wu, Ph.D
        \begin{innerlist}
        \item Built a web service search engine for testing several service discovery algorithms
        \item Led a team working on a news radio app based on recommender system
        \item Implemented and measured several service computing-related algorithms 
        \end{innerlist}
\end{outerlist}

\vspace{.1in}

\textbf{Visiting Student} \hfill {May 2013 to Oct 2013}
\begin{outerlist}
\item[] Computer Science Department,\\
        University of California, Santa Barbara\\
        Supervisor: \href{http://www.cs.ucsb.edu/~xyan/}
                   {Xifeng Yan, Ph.D}
        \begin{innerlist}
        \item Built the interface and querying language parser for SLQ, the graph querying system
        \item Developed an open information extraction tool for parsing nested and complex sentences, outperforming all other extractors on our corpus
        \end{innerlist}
\end{outerlist}

\section{Publications}
\vspace{-.1275in}
\begin{bibsection}
    \item Shengqi Yang, {\bf Yanan Xie}, Yinghui Wu, Tianyu Wu, Huan Sun, Jian Wu and Xifeng Yan.  ``SLQ: A User-friendly Graph Querying System." \emph{ACM SIGMOD Conference on Management of Data (SIGMOD 2014 Demo)}, Snowbird, Utah, USA. June 22 - 27, 2014.
    \item {\bf Yanan Xie}, Liang Chen, Kunyang Jia, Lichuan Ji, Jian Wu.  ``iNewsBox: Modeling and Exploiting Implicit Feedback for Building Personalized News Radio." \emph{ACM Conference of Information and Knowledge Management (CIKM 2013 Demo)}, Burlingame, CA, USA. Oct 27 - Nov 1, 2013.
    \item Li Kuang, Liang Chen, {\bf Yanan Xie}, and Jian Wu.  ``Full Recognition of Massive Commodities Based on Property Set." \emph{IEEE International Congress on Big Data (BigData Congress 2013)}, Santa Clara, CA, USA. June 27 - July 2, 2013.
    \item Jian Wu, Liang Chen, {\bf Yanan Xie}, Lichuan Ji, Zhaohui Wu.  ``Modeling and Exploring Historical Records to Facilitate Service Composition." \emph{International Journal of Web and Grid Services (IJWGS)}(accepted in April 2013).
    \item Hanze Xu, {\bf Yanan Xie}, Dinglong Duan, Liang Chen, Jian Wu.  ``FTCRank: Ranking Components for Building Highly Reliable Cloud Applications." \emph{17th Pacific-Asia Conference on Knowledge Discovery and Data Mining (PAKDD 2013 Workshop)}, Gold Coast, Australia. April 14 - 17, 2013.
    \item Jian Wu, Liang Chen, {\bf Yanan Xie}, Zibin Zheng. ``Titan: A System for Effective Web Service Discovery."  \emph{21th International World Wide Web Conference (WWW 2012 Demo)}, Lyon, France. April 16-20, 2012.
        %To appear \emph{Alcoholism: Clinical and Experimental Research}, 2012.
    \item {\bf Yanan Xie}. ``An AJAX-based Method for Implementing Differential Transmission of Desktop Image." Chinese Patent 200910043056, 2012.
\end{bibsection}

\section{Awards}
\begin{innerlist}
\item China Computer Federation Outstanding Undergraduate (99 in China) \hfill 2013
\item He Zhijun Scholarship \hfill 2013\\
(1\% in College of Cmputer Science and Technology, Zhejiang Univ.)
\item CKC College Scholarship for Excellence in Research and Innovation\hfill 2013\\
(1\% in Chukochen Honors College) 
\item First-class Scholarship for Excellence in Research and Innovation\hfill 2012\\
(1\% in Zhejiang Univ.)
\item Third Prize for ACM Programming Contest in Zhejiang Province \hfill 2011
\item Second Prize for Challenge Cup Contest in Zhejiang University \hfill 2011
\item Second Prize, Awarding Program for Future Scientists \hfill 2009
\item First Prize, National Olympiad in Informatics in Provinces \hfill 2008, 2009\\
 China Computer Federation 
\end{innerlist}

\section{Conference Presentations}
\begin{innerlist}
\item 21th International World Wide Web Conference, Lyon, France \hfill Apr 2012
\item 17th Pacific-Asia Conference on Knowledge Discovery and Data Mining \hfill Apr 2013\\
Gold Coast, Australia
\item IEEE International Congress on Big Data, Santa Clara, CA, USA\hfill June, 2013
\end{innerlist}

\section{Service}

Vice president of China Computer Federation student chapter \hfill {Jan 2013 -- June 2014}\\
 at Zhejiang University
\begin{innerlist}
    \item Organized several professional lectures towards students
    \item Hosted a weekly data mining and machine learning seminar
\end{innerlist}

\section{Professional Skills}

Computer Programming:

\begin{innerlist}
    \item ASP.NET, C, C$+$$+$, Objective-C, Java, JavaScript, Pascal(Delphi), PHP, Python, 
        UNIX shell scripting, SQL, Swift
\end{innerlist}

\section{References}

Jian Wu
\begin{innerlist}
\item[] Professor \hfill {Phone: +86-13819170001}\\
College of Computer Science \& Technology \hfill{E-mail: wujian2000@zju.edu.cn}\\
Zhejiang University
\end{innerlist}

\halfblankline

Xifeng Yan
\begin{innerlist}
\item[] Professor \hfill {Phone: +1(805)699-6018}\\
Computer Science Department \hfill{E-mail: xyan@cs.ucsb.edu }\\
University of California, Santa Barbara
\end{innerlist}


\end{document}

%%%%%%%%%%%%%%%%%%%%%%%%%% End CV Document %%%%%%%%%%%%%%%%%%%%%%%%%%%%%

%----------------------------------------------------------------------%
% The following is copyright and licensing information for
% redistribution of this LaTeX source code; it also includes a liability
% statement. If this source code is not being redistributed to others,
% it may be omitted. It has no effect on the function of the above code.
%----------------------------------------------------------------------%
% Copyright (c) 2007, 2008, 2009, 2010, 2011 by Theodore P. Pavlic
%
% Unless otherwise expressly stated, this work is licensed under the
% Creative Commons Attribution-Noncommercial 3.0 United States License. To
% view a copy of this license, visit
% http://creativecommons.org/licenses/by-nc/3.0/us/ or send a letter to
% Creative Commons, 171 Second Street, Suite 300, San Francisco,
% California, 94105, USA.
%
% THE SOFTWARE IS PROVIDED "AS IS", WITHOUT WARRANTY OF ANY KIND, EXPRESS
% OR IMPLIED, INCLUDING BUT NOT LIMITED TO THE WARRANTIES OF
% MERCHANTABILITY, FITNESS FOR A PARTICULAR PURPOSE AND NONINFRINGEMENT.
% IN NO EVENT SHALL THE AUTHORS OR COPYRIGHT HOLDERS BE LIABLE FOR ANY
% CLAIM, DAMAGES OR OTHER LIABILITY, WHETHER IN AN ACTION OF CONTRACT,
% TORT OR OTHERWISE, ARISING FROM, OUT OF OR IN CONNECTION WITH THE
% SOFTWARE OR THE USE OR OTHER DEALINGS IN THE SOFTWARE.
%----------------------------------------------------------------------%
